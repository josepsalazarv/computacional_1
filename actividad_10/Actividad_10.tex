\documentclass[a4paper]{article}

\usepackage[english]{babel}
\usepackage[utf8]{inputenc}
\usepackage{amsmath}
\usepackage{graphicx}
\usepackage[colorinlistoftodos]{todonotes}

\title{Actividad 10}

\author{Jose Pablo Salazar Velazquez}

\date{abril del 2016}

\begin{document}
\maketitle

\section{Introduccion}

La animación ha sido una herramienta impresindible para la comunidad científica. Nos ayuda a visualizar modelos de moviemiento de cuerpos o recrear experimentos, y así nos facilita la tarea a la hora de comprender un fenómeno. Matplotlib, de la biblioteca de Python, nos permite realizar estas animaciones.

En esta practica, haremos animaciones, similares a las 8 que aparecen en el articulo de wikipedia, repruduciendo el espacio fase y físico de los pendulos. 
El codigo para las animaciones fue tomado de internet en Matplotlb. Se adapto para resolver resolver el ejercicio.

\section{Pendulo}
\label{sec:examples}

A continuacion, se presentará el codigo base que te utilizo para correr las simulaciones.

\begin{verbatim}
from numpy import sin, cos
import numpy as np
import matplotlib.pyplot as plt
import scipy.integrate as integrate
import matplotlib.animation as an
from matplotlib.lines import Line2D
from scipy.integrate import odeint

class DoublePendulum:
    def __init__(self,
                 init_state = [0, 0, 0, 0],
                 L1 = 1.0,  # Longitud del pendulo 1
                 L2 = 0.0,  # Longitud del pendulo 2
                 M1 = 1.0,  # Masa del pendulo 1
                 M2 = 1.0,  # Masa del pendulo 2
                 G = 9.8,   # Aceleracion por la gravedad
                 origin=(0, 0)): 
        self.init_state = np.asarray(init_state, dtype='float')
        self.params = (L1, L2, M1, M2, G)
        self.origin = origin
        self.time_elapsed = 0

        self.state = self.init_state * np.pi / 180.
    
    def position(self):
        (L1, L2, M1, M2, G) = self.params

        x = np.cumsum([self.origin[0],
                       L1 * sin(self.state[0]),
                       L2 * sin(self.state[2])])
        y = np.cumsum([self.origin[1],
                       -L1 * cos(self.state[0]),
                       -L2 * cos(self.state[2])])
        return (x, y)

    def energy(self):
        (L1, L2, M1, M2, G) = self.params

        x = np.cumsum([L1 * sin(self.state[0]),
                       L2 * sin(self.state[2])])
        y = np.cumsum([-L1 * cos(self.state[0]),
                       -L2 * cos(self.state[2])])
        vx = np.cumsum([L1 * self.state[1] * cos(self.state[0]),
                        L2 * self.state[3] * cos(self.state[2])])
        vy = np.cumsum([L1 * self.state[1] * sin(self.state[0]),
                        L2 * self.state[3] * sin(self.state[2])])

        U = G * (M1 * y[0] + M2 * y[1])
        K = 0.5 * (M1 * np.dot(vx, vx) + M2 * np.dot(vy, vy))

        return U + K

    def dstate_dt(self, state, t):
        (M1, M2, L1, L2, G) = self.params

        dydx = np.zeros_like(state)
        dydx[0] = state[1]
        dydx[2] = state[3]

        cos_delta = cos(state[2] - state[0])
        sin_delta = sin(state[2] - state[0])

        den1 = (M1 + M2) * L1 - M2 * L1 * cos_delta * cos_delta
        dydx[1] = (M2 * L1 * state[1] * state[1] * sin_delta * cos_delta
                   + M2 * G * sin(state[2]) * cos_delta
                   + M2 * L2 * state[3] * state[3] * sin_delta
                   - (M1 + M2) * G * sin(state[0])) / den1

        den2 = (L2 / L1) * den1
        dydx[3] = (-M2 * L2 * state[3] * state[3] * sin_delta * cos_delta
                   + (M1 + M2) * G * sin(state[0]) * cos_delta
                   - (M1 + M2) * L1 * state[1] * state[1] * sin_delta
                   - (M1 + M2) * G * sin(state[2])) / den2
        
        return dydx

    def step(self, dt):
        self.state = integrate.odeint(self.dstate_dt, self.state, [0, dt])[1]
        self.time_elapsed += dt

#-----------------------
#CI para el pendulo
theta0=190
v0= 20
pendulum = DoublePendulum([theta0, v0, 0.0, 0.0])

#-----------------------
#CI para el espacio fase
g = 9.81 #valor de g
l = 1.0 #longitud
b = 0.0 #no friccion
c = g/l

X_f1 =np.array([(theta0/180.0)*np.pi,(v0/180.0)*np.pi])
t = np.linspace(0,30,500)

#ED del pendulo
def p (y, t, b, c):
    theta, omega = y
    dy_dt = [omega,-b*omega -c*np.sin(theta)]
    return dy_dt

#Trayectoria
y0 = X_f1                       
X = odeint(p, y0, t, args=(b,c))         

#-----------------------
#Animacion del pendulo
dt = 1./60.
fig = plt.figure()
ax = fig.add_subplot(111, aspect='equal', autoscale_on=False,
                     xlim=(-2, 2), ylim=(-2, 2))
ax.grid()

line, = ax.plot([], [], 'o-', lw=2, color='r')
time_text = ax.text(0.02, 0.95, '', transform=ax.transAxes)
energy_text = ax.text(0.02, 0.90, '', transform=ax.transAxes)

def init():
    #iniciando animacion
    line.set_data([], [])
    time_text.set_text('')
    energy_text.set_text('')
    return line, time_text, energy_text

def animate(i):
    #animar las fotitos
    global pendulum, dt
    pendulum.step(dt)
    line.set_data(*pendulum.position())
    return line, time_text, energy_text

from time import time
t0 = time()
animate(0)
t1 = time()
interval = 1000 * dt - (t1 - t0)

ani = an.FuncAnimation(fig, animate, frames=300,
                              interval=interval, blit=True, init_func=init)
plt.show()

class SubplotAnimation(an.TimedAnimation):
    def __init__(self):
        fig = plt.figure()
        ax1 = fig.add_subplot(1, 1, 1)
       
        self.t = np.linspace(0, 80, 400)
        self.x = X[:,0]
        self.y = X[:,1]

        self.line1 = Line2D([], [], color='y')
        self.line1a = Line2D([], [], color='g', linewidth=2)
        self.line1e = Line2D(
            [], [], color='g', marker='o', markeredgecolor='r')
        ax1.add_line(self.line1)
        ax1.add_line(self.line1a)
        ax1.add_line(self.line1e)
        ax1.set_xlim(-10, 10)
        ax1.set_ylim(-10, 10)
        ax1.grid()
        ax1.set_aspect('equal', 'datalim')

        an.TimedAnimation.__init__(self, fig, interval=50, blit=True)

    def _draw_frame(self, framedata):
        i = framedata
        head = i - 1
        head_slice = (self.t > self.t[i] - 1.0) & (self.t < self.t[i])

        self.line1.set_data(self.x[:i], self.y[:i])
        self.line1a.set_data(self.x[head_slice], self.y[head_slice])
        self.line1e.set_data(self.x[head], self.y[head])

    def new_frame_seq(self):
        return iter(range(self.t.size))

    def _init_draw(self):
        lines = [self.line1, self.line1a, self.line1e]
        for l in lines:
            l.set_data([], [])

ani = SubplotAnimation()
plt.show()
\end{verbatim}
\section{Conclusión}
Las animaciones ayudan a ver la trayectoria del pendulo, sin necesidad de estarlo observando en persona.
Se logró el objetivo de la práctica. 

\end{document}