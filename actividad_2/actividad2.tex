\documentclass[12pt]{article}
\usepackage{mathtools}
\usepackage[utf8]{inputenc}
\usepackage[english]{babel}
\usepackage{graphicx}
\usepackage{graphicx}
\usepackage{amsmath}
\usepackage{moreverb}
\usepackage{array}
%\graphicspath{ {\Users\Jose\Desktop\COMPUTACIONAL} }
\DeclareGraphicsExtensions{.png}
\title{Péndulo simple}
\author{Jose Pablo Salazar Velazquez}
\date{}
\usepackage{graphicx}
\begin{document}
\maketitle

\section{Introducción}
El lenguaje de programación Python es uno de los más usados en el mundo, según una medición de TIOBE Programming Community Index (2008) es el octavo más popular. Además es el tercero más popular en aquellos lenguajes que no basan su sintaxis gramatical en C. Un estudio mostró que Python hace un uso de la memoria mejor que JAVA y no tan lejos de la eficiencia de C o C++. Grandes organizaciones utilizan Python para algunos de sus productos como Google, Yahoo!, CERN, NASA, etc. También es utilizado en la computación científica gracias a librerías como NumPy, SciPy y Matplotlib. Es empleado en tareas de inteligencia artificial, como por ejemplo en tareas de procesamiento de lenguajes naturales.
\pagebreak
\subsection{Problema 1}
Se pide encontrar el tiempo que tarda una pelota en llegar al suelo, cuando es soltada desde una altura "y", en metros, ingresada por el usario. \\
Basandonos en el codigo originalmente proporcionado: \\
\begin{boxedverbatim}
h = float(input("Proporciona la altura de la torre: "))
t = float(input("Ingresa el tiempo: "))
s = 0.5*9.81*t**2
print("La altura de la pelota es", h-s, "metros")
\end{boxedverbatim} 
\\
Se hicieron los calculos, para saber el tiempo que tarda en tocar el suelo, cuando se deja caer de una altura de 10m \\
\begin{boxedverbatim}
#importo la funcion sqrt de math
from math import sqrt

#pido altura al usuario
h = float(input("Proporciona la altura en metros de la torre: "))
g=9.81
t=sqrt(2*h/g) #Calculo de tiempo de caida

#Impresion de Resultados
print ("El tiempo de caida es:", t ,"s")
\end{boxedverbatim}
\\
Se obtuvo, que para una altura de 10m, tardaria 1.42s en llegar al suelo.
%%%%%%%%%%%%%%%%%%%%%%%%%%%%%%%%%%%%%%%%%%%%%%%%%%%%%%%%%%%%%%%%%%%%%%%%
\pagebreak
\subsection{Problema 2}
Para el segundo problema, se pide encontrar la altura de un satelite, si conocemos su periodo: \\
\begin{boxedverbatim}
from math import pi
G=6.67e-11
m=5.97e24
r=6371000
t= float(input("ingresa el periodo del satélite:"))
T= t*60
a=(g*m*t*t) / (4*pi*pi)
b=a**(1./3.)
Y=b-r
print ("La altura del satélite es", h, "metros.")
\end{boxedverbatim}
%%%%%%%%%%%%%%%%%%%%%%%%%%%%%%%%%%%%%%%%%%%%%%%%%%%%%%%%%%%%%%%%%%%%%%%
\\
\pagebreak
\subsection{Problema 3}
Analizamos el codigo proporcionado para encontrar las coordenadas polares de un punto: \\
\begin{boxedverbatim}
from math import sin,cos,pi
r = float(input("Introduce r: "))
d = float(input("Ingresa theta en grados: "))
theta = d*pi/180
x = r*cos(theta)
y = r*sin(theta)
print("x =",x," y =",y) 
\end{boxedverbatim}
  \\
 Se modificó para poder encontrar las coordenadas esfericas de un punto dado:
 from math import pi, sin, atan, acos, sqrt \\
 \begin{boxedverbatim}
print ("Especifica las coordenadas cartesianas del punto:")
x = float(input("Introduce x: "))
y = float(input("Introduce y: "))
z = float(input("Inytoduce z: "))
r = sqrt(x**2 + y**2 + z**2)
theta = acos(z/r)
phi = atan(y/x)
print ("Las coordenas esféricas son:")
print("r =",r,"theta =",theta,"phi =",phi)
\end{boxedverbatim}
\\
Se obtuvo que:\\
\begin{boxedverbatim}
Especifica las coordenadas cartesianas del punto:
Introduce x: 8
Introduce y: 9
Inytoduce z: 10
Las coordenas esféricas son:
r = 15.652475842498529 theta = 0.8777592647425649 phi = 0.844153986113171
\end{boxedverbatim}
\pagebreak
\subsection{Problema 4}
Usando la logica de los numero de fibonacci, se encontraron los numeros de Catalán, usando el codigo: \\
\begin{boxedverbatim}
n,C1,C2 = 0,1,1
while(C2 < 1000000): 
    print(C2)
    C2= C1*(4*n+2)/(n+2)
    n=n+1
    C1=C2
 \end{boxedverbatim}
 \\
 De el codido anterior, se obtuvieron los numeros de catalán, menores a 10000000 :
  \\
 \begin{boxedverbatim}
 1
1.0
2.0
5.0
14.0
42.0
132.0
429.0
1430.0
4862.0
16796.0
58786.0
208012.0
742900.0
 \end{boxedverbatim}
\end{document}